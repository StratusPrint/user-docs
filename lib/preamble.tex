\documentclass [10pt]{book}
\oddsidemargin=0.0cm
\textwidth=16.5cm
\textheight=22.8cm
\topmargin=-1.0cm
\usepackage[utf8]{inputenc}
\usepackage{booktabs}
\usepackage{fancyhdr}
\usepackage{multicol}
\usepackage{mathtools}% Loads amsmath
\usepackage{flowchart}
\usepackage{xcolor}
\usepackage{lipsum}
\usepackage{graphicx}
\usepackage{float}
\usepackage{geometry}
\renewcommand*\contentsname{Contents}
\definecolor{namecolor}{cmyk}{1,.50,0,.10}
\definecolor{titlepagecolor}{cmyk}{1,.209,0,.831}
\renewcommand{\thesection}{\arabic{section}}
\renewcommand{\headrulewidth}{3pt}% 2pt header rule
\renewcommand{\headrule}{\hbox to\headwidth{%
  \color{titlepagecolor}\leaders\hrule height \headrulewidth\hfill}}
\pagestyle{fancy}
\fancyhf{}
\fancyhead[LE]{\hspace*{-0.2\headwidth}\colorbox{titlepagecolor!20}{\makebox[\dimexpr0.2\headwidth-2\fboxsep][c]{\strut\thepage}}%
               \colorbox{titlepagecolor}{\makebox[\dimexpr\headwidth-2\fboxsep][l]{Section \strut\thesection}}}
\fancyhead[LO]{\colorbox{titlepagecolor}{\makebox[\dimexpr\headwidth-2\fboxsep][r]{Section \strut\thesection}}%
                \colorbox{titlepagecolor!20}{\makebox[\dimexpr0.2\headwidth-2\fboxsep][c]{\strut\thepage}}}
\renewcommand{\headrulewidth}{0pt}
%\lhead{\fancyplain{}{\leftmark }}
%\rhead{\fancyplain{}{Alarm Clock}}
%\renewcommand{\headrulewidth}{1pt}
\lfoot{}
\usepackage{listings}
\usepackage{multicol}

% the following is needed for syntax highlighting
\usepackage{color}
\definecolor{dkgreen}{rgb}{0,0.6,0}
\definecolor{gray}{rgb}{0.5,0.5,0.5}
\definecolor{mauve}{rgb}{0.58,0,0.82}
\lstset{ %
language=[C]{Assembler},      % the language of the code
basicstyle=\scriptsize,         % the size of the fonts that are used for the code
numbers=left,                   % where to put the line-numbers
numberstyle=\tiny\color{gray},  % the style that is used for the line-numbers
stepnumber=1,                   % the step between two line-numbers. If it's 1, each line
% will be numbered
numbersep=5pt,                  % how far the line-numbers are from the code
backgroundcolor=\color{white},  % choose the background color. You must add \usepackage{color}
showspaces=false,               % show spaces adding particular underscores
showstringspaces=false,         % underline spaces within strings
showtabs=false,                 % show tabs within strings adding particular underscores
frame=single,                   % adds a frame around the code
rulecolor=\color{white},        % if not set, the frame-color may be changed on line-breaks within not-black text (e.g. commens (green here))
tabsize=2,                      % sets default tabsize to 2 spaces
captionpos=b,                   % sets the caption-position to bottom
breaklines=true,                % sets automatic line breaking
breakatwhitespace=false,        % sets if automatic breaks should only happen at whitespace
title=\lstname,                 % show the filename of files included with \lstinputlisting;
% also try caption instead of title
keywordstyle=\color{blue},          % keyword style
commentstyle=\color{gray},       % comment style
stringstyle=\color{mauve},         % string literal style
escapeinside={\%*}{*)},            % if you want to add a comment within your code
morekeywords={*,...}               % if you want to add more keywords to the set
}
% facilitates the creation of memory maps. Start address at the bottom, end address at the top.
% Addresses will be print with a leading '0x' and in upper case.
% syntax: \memsection{end address}{start address}{height in lines}{text in box}
\newcommand{\memsection}[4]{
  \bytefieldsetup{bitheight=#3\baselineskip}	% define the height of the memsection
  \bitbox[]{8}{
    \texttt{0x\uppercase{#1}}	 % print end address
    \\ \vspace{#3\baselineskip} \vspace{-2\baselineskip} \vspace{-#3pt} % do some spacing
    \texttt{0x\uppercase{#2}} % print start address
  }
  \bitbox{16}{#4} % print box with caption
}